\documentclass[12pt,letterpaper,fleqn]{article}
\usepackage{amsmath}
\usepackage{graphicx}
\usepackage{refstyle}
\usepackage{calc}
\usepackage[total={7in,10in},top=0.25in,left=0.75in,includehead]{geometry}
\graphicspath{{\Users\kaya\Dropbox\PictorialSRC\Math278}}
\DeclareGraphicsExtensions{.ps,.JPG}

\begin{document}
\framebox{\begin{minipage}[t]{\widthof{ May 10th 2014}}
Kaya Ota\\
Rob Komas\\ 
Math 278\\
May 10th 2014
\end{minipage}}\hfill
\begin{center}
{\LARGE\bf Mass Spring Problem}
\end{center}
\section{Question}
We are going to solve the differential equation that describes its motion of spring.
Let t is any time and dependent variable. Also, if $x < 0$, then a spring is compressed and if $x > 0$, then a spring is stretched from its equilibrium position.
The differential equation we are going to solve is $x''(t) + 4x(t) = f(t)$ where 
\[	f(t) = \begin{cases}
		0 &  t < 5\\
		k &  5 \leq t < 5.1\\
		0 &  t \geq 5.1
		\end{cases}
\]
\section{Rewriting Unit Step Function}
\[
	f(t) = \begin{cases}
		0 &  t < 5\\
		k &  5 \leq t < 5.1\\
		0 &  t \geq 5.1
		\end{cases}
\]
The first condition tells a unit function behaves as:\\
\begin{figure}[htbp]
	\begin{minipage}{0.5\hsize}
		\begin{center}
		\includegraphics[width = 10cm, height = 6cm, angle = 0]{/Users/kaya/Dropbox/PictorialSRC/Math278/MassSpringPicts/Slide1.jpg}
		\caption{unit step function $t < 5$}
		\end{center}
	\end{minipage}
	\begin{minipage}{0.5\hsize}
		\begin{center}
			\includegraphics[width = 10cm, height = 6cm, angle = 0]{/Users/kaya/Dropbox/PictorialSRC/Math278/MassSpringPicts/Slide2.jpg}
		\caption{combination of unite step functions}
		\end{center}
	\end{minipage}
\end{figure}\\
To obtain the green function, the purple line needs to be zero at t = 5.1.\\
So, the subtract k at t = 5.1 as unit step function lined in purple.
Now, we enable to rewrite the form of given $ f(t) $ is:
$$f(t) = ku(t-5.0)-ku(t-5.1) $$ in green.
Then, we can rewrite the given differential equation as:
$$x''(t) + 4x'(t) = ku(t-5.0)-ku(t-5.1)$$
where u(t) is unit step function.\\
\section{Solve the DE using Laplace Transform}
Take a Laplace transform for both sides.
$$\mathcal{L}[x''(t) ] +\mathcal{L}[4x'(t)] = \mathcal{L}[ku(t-5.0)-ku(t-5.1)] $$
By the properties of Linearity of Laplace transform,\\
(closed under vector addition and scalar multiplication)
$$\mathcal{L}[x''(t)] + 4\mathcal{L}[x'(t)] = k\mathcal{L}[u(t-5.0)] - k\mathcal{L}[u(t-5.1)]$$
By the property of differentiation of Laplace, 
$$ s^2\mathcal{L}[x]-C_0s-C_1+4\mathcal{L}[x] = \frac{k}{s}\mathrm{e}^{-5s}-\frac{k}{s}\mathrm{e}^{-5.1s}  $$ 
On the left side, factor out by $ \mathcal{L}[x]$ to obtain:
\begin{eqnarray*}
(s^2 + 4)\mathcal{L}[x] &=& \frac{k}{s}\mathrm{e}^{-5s}-\frac{k}{s}\mathrm{e}^{-5.1s}+C_0s+C_1\\
\mathcal{L}[x] &=& \frac{1}{s^2 + 4}(\frac{k}{s}\mathrm{e}^{-5s}-\frac{k}{s}\mathrm{e}^{-5.1s}+C_0s+C_1)\\
&=&\frac{1}{(s^2 + 4)}\frac{k}{s}\mathrm{e}^{-5s}-\frac{1}{(s^2 + 4)}\frac{k}{s}\mathrm{e}^{-5.1s}+\frac{1}{s^2 + 4}C_0s+\frac{1}{s^2 + 4}C_1
\end{eqnarray*}\\
To find out x(t), now we simply need to take inverse Laplace transformation.
$$\mathcal{L}^{-1}[x] = \mathcal{L}^{-1}[\frac{1}{(s^2 + 4)}\frac{k}{s}\mathrm{e}^{-5s}]-\mathcal{L}^{-1}[\frac{1}{(s^2 + 4)}\frac{k}{s}\mathrm{e}^{-5.1s}]+\mathcal{L}^{-1}[\frac{1}{s^2 + 4}C_0s]+\mathcal{L}^{-1}[\frac{1}{s^2 + 4}C_1]$$
Laplace transforms we are going to use are:
\begin{eqnarray*}
\mathcal{L}[\sin(ax)] &=& \frac{a}{(s^2+a^2)}\\
\mathcal{L}[\cos(ax)] &=& \frac{s}{(s^2+a^2)}\\
\mathcal{L}[u(t-c)] &=& \frac{1}{s}\mathrm{e}^{-cs}\\
\mathcal{L}[\sin^{2}(ax)] &=& \frac{2a^2}{s(s^2+4a^2)} \Rightarrow \frac{1}{2}\mathcal{L}[sin^{2}(ax)] = \frac{a^2}{s(s^2+a^2)}
\end{eqnarray*}
So, each terms are transformed as:\\
\begin{eqnarray*}
k\mathcal{L}^{-1}[\frac{1}{s(s^2+4)}\mathrm{e}^{-5.0s}] &=& \frac{k}{2}\sin^{2}(t-5.0)u(t-5.0)\\
k\mathcal{L}^{-1}[\frac{1}{s(s^2+4)}\mathrm{e}^{-5.1s}] &=& \frac{k}{2}\sin^{2}(t-5.1)u(t-5.1)\\
C_0\mathcal{L}^{-1}[\frac{s}{s^2+4}] &=& C_0\cos(2t)\\
C_1\mathcal{L}^{-1}[\frac{1}{s^2+4}] &=& C_1\sin(2t)\\
\end{eqnarray*}
Because Laplace transform is a linear transform, we can add up each terms to find the original x that we look for.\\
Therefore,
\begin{eqnarray*}
\mathcal{L}^{-1}[x] &=& \mathcal{L}^{-1}[\frac{1}{(s^2 + 4)}\frac{k}{s}\mathrm{e}^{-5s}]-\mathcal{L}^{-1}[\frac{1}{(s^2 + 4)}\frac{k}{s}\mathrm{e}^{-5.1s}]+\mathcal{L}^{-1}[\frac{1}{s^2 + 4}C_0s]+\mathcal{L}^{-1}[\frac{1}{s^2 + 4}C_1]\\
x(t) &=& \frac{k}{2}\sin^{2}(t-5.0)u(t--5.0)-\frac{k}{2}\sin^{2}(t-5.1)u(t-5.1)+C_0\cos(2t)+\frac{C_1}{2}\sin(2t)
\end{eqnarray*}
\section{Check the consistency}
If $ t < 5 $, then $x(t) = \frac{k}{2}\sin^{2}(t-5.0)u(t--5.0)-\frac{k}{2}\sin^{2}(t-5.1)u(t-5.1)+C_0\cos(2t)+\frac{C_1}{2}\sin(2t)$ turns out $x(t) = C_0\cos(2t)+\frac{C_1}{2}\sin(st)$
\begin{eqnarray}
x(t)  &=& C_0\cos(2t)+\frac{C_1}{2}\sin(2t)\\
x'(t)  &=& -2C_0\sin(2t)+C_1\cos(2t)\\
x''(t)  &=& -4C_0\cos(2t)-2C_1\sin(2t)
\end{eqnarray}
To check the consistency, we apply (1) and (3) to the given differential equation.
$$x''(t)+4x(t) =  -4C_0\cos(2t)-2C_1\sin(2t) + 4C_0\cos(2t)+2C_1\sin(2t) = 0$$
The given unit step function clams that $f(t) = 0$ for $ t < 5 $\\
So, the differential equation is satisfied. \\\\
If $ t \geq 5 $, then $x(t) = \frac{k}{2}\sin^{2}(t-5.0)u(t--5.0)-\frac{k}{2}\sin^{2}(t-5.1)u(t-5.1)+C_0\cos(2t)+\frac{C_1}{2}\sin(2t)$ turns out to be $x(t) = \frac{k}{2}\sin^{2}(t-5.0)+C_0\cos(2t)+\frac{C_1}{2}\sin(2t)$
\begin{eqnarray}
x(t) &=& \frac{k}{2}\sin^{2}(t-5.0)+C_0\cos(2t)+\frac{C_1}{2}\sin(2t)\\
x'(t) &=& -\frac{k}{2}\sin(10-2t)+C_0cos(2t)+\frac{C_1}{2}sin(2t)\\
x''(t) &=& k\cos(10-2t)-4C_0\cos(2t)-2C_2sin(2t)
\end{eqnarray}
To check the consistency, we apply (4) and (6) to the given differential equation.\\
\begin{eqnarray*}
x''(t) + 4x(t) &=& k\cos(10-2t)-4C_0\cos(2t)-2C_1\sin(2t)+4\frac{k}{2}\sin^{2}(t-5)+4C_0\cos(2t)+2C_1\sin(2t)\\
 &=& k\cos(10-2t)+2ksin^{2}(t-5)\\
 &=&k(1) \\
 &=&k 
\end{eqnarray*} 
$x''(t) + 4x(t) = k $ is true for $t \geq 5 $ and $t < 5.1$ \\
If $t \geq 5.1$,then both unit functions $u(t-5.0)$ and $u(t-5.1)$ are 1.\\
We know $x''(t) + 4x(t) = k $ if $x = \frac{k}{2}\sin^2(t-5)$.\\
That result, k, does not depend on t and 5, so we can assume $x''(t) + 4x(t) = k $ if $x = \frac{k}{2}\sin^2(t-5.1)$\\
Then,
\begin{eqnarray*}
x''(t) + 4x(t) &=& \frac{k}{2}\sin^2(t-5) - \frac{k}{2}\sin^2(t-5.1)\\
&=& k - k \\
&=& 0
\end{eqnarray*}
$x''(t) + 4x(t) = 0$ is true for $t \geq 5.1 $\\
Therefore, the solution we get is satisfied for all conditions.\\
Hence, the solution is right.
\section{Graphs}
\begin{figure}[h]
\begin{center}
\includegraphics[width = 15cm,height=10cm,angle =0]{/Users/kaya/Dropbox/PictorialSRC/Math278/MassSpringPicts/MassSpring5.jpg}
\end{center}
\end{figure}
As actual spring does, this function starts osculating since time reaches to 5.\\
This function changes its behavior before t = 5, between t = 5 and t = 5.1, and after t = 5.1 \\
From the previous graph, we know the spring osculates from t =5.\\
So, now we zoom up between t = 5 and t = 5.1\\
\begin{figure}[htbp]
\begin{center}
\includegraphics[width = 10cm, height = 7cm, angle = 0]{/Users/kaya/Dropbox/PictorialSRC/Math278/MassSpringPicts/MotionSpringIncrease.jpg}
\end{center}
\end{figure}\\
Interestingly, the function increases smoothly for $5 \leq t < 5.1$ 
Because this function increase smoothly, which means continuous here, we can differentiate even though the original right hand side contains unit step.
\section{Closure}
To show that the solution is right, I tried to solve the differential equation by Eula's method, however, it have not worked well.









\end{document}



